\documentclass[addpoints]{exam}
\usepackage{charter}

\begin{document}
\header{Physics 462}{Midterm I}{12 February 2009}
\coverchead{\Large Physics 462 Midterm I}
\cellwidth{4em}
\gradetablestretch{3.0}
\headrule

\begin{coverpages}
\vskip0.1cm\hrule\vskip0.5cm
\makebox[\textwidth]{Name:\enspace\hfill}
\vskip 1cm
You may use your homework portfolio as a reference with this exam.
\vskip 2cm
\begin{center}
\gradetable
\end{center}
\end{coverpages}

\begin{questions}
  \question {\bf Expectation Values}
  \begin{parts}
    \part[9] Compute $\langle T\rangle$ (Kinetic energy; transmission
    coefficient wouldn't make sense in this context) for the second
    excited state of the infinite square well.\vspace{\fill}
    \part[9] Compute $\langle T\rangle$ for the second excited state of the harmonic
    oscillator.\vspace{\fill}
    \part[7] Draw a potential for which $\langle x\rangle\neq 0$.\vspace{\fill}
  \end{parts}
  \newpage\question {\bf Downward step potential} Consider the
  downward step potential (as in problem 2.35).
  \begin{parts}
    \part[5] Compute the reflectance coefficient $R$ for an energy
    just slightly above the top of the step.  That is, $E = \epsilon$,
    where $\epsilon$ is some very small number, and the step goes from
    $V=0$ on the left down to $V=-V_0$ on the right.\vspace{\fill}
    \part[7] Give a physical justification for your answer in part
    (a).\vspace{\fill}
    \part[5] Compute the reflectance coefficient $R$ for $E \gg
    V_0$.\vspace{\fill}
    \part[8] Give a physical justification for your answer in part
    (c).\vspace{\fill}
  \end{parts}
  \newpage\question {\bf Triple delta function potential} Note that in
  this problem I am not asking you to perform an entirely new
  calculation; rather, I would like you to generalize a result you
  obtained in homework.
  \begin{parts}
    \part[15] Based on your results for the double delta function
    potential, sketch the wave functions for the bound states of the
    triple delta function potential $V = -\alpha\left(\delta(x+a) +
    \delta(x) + \delta(x-a)\right)$.\vspace{\fill}
    \part[5] Which of these is the ground state, and why?\vspace{\fill}
    \part[5] Explain the parity (evenness or oddness)
    of these solutions.\vspace{\fill}
  \end{parts}
  \newpage\question {\bf Linear combination} A particle in a harmonic
  oscillator potential has as its initial wave function a mixture of
  the first two stationary states: \[ \Psi(x,0) = A\left[\psi_0(x) +
    \psi_1(x)\right] \]
  \begin{parts}
    \part[8] Normalize $\Psi(x,0)$.\vspace{\fill}
    \part[8] Find $\Psi(x,t)$ and $|\Psi(x,t)|^2$.\vspace{\fill}
    \part[9] Compute $\langle x\rangle$. (Remember that there is a trick to
    make this easy) \vspace{\fill}
  \end{parts}
  
\end{questions}

\end{document}