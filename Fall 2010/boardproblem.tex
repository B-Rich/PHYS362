\documentclass{article}
\usepackage{fullpage}
\usepackage{charter}


\begin{document}
\centerline{\Large How to work a problem at the board}
\vskip0.1cm\hrule\vskip0.5cm

Over the course of the semester, you will have several opportunities
to lead a discussion of a particular homework problem at the board.
Your performance will be graded according to the attached rubric, but
some additional notes may help you as you prepare for these
opportunities.

\begin{itemize}
\item {\bf Always start the problem with a reference.}  Tell us which
  problem you're solving.  This may seem unnecessary; we're all in the
  same class, after all, and ought to know what's going on.  It is
  sometimes the case, however, that one or more of us will be
  distracted enough before or at the beginning of class that this
  small step will be appreciated.
\item {\bf Always give a synopsis of the problem.}  This doesn't
  mean you should always read the problem text as given, but
  you do need to give us enough information to remind us what
  we're looking at.
\item {\bf Prepare.}  In addition to working the problem before coming
  to class, you should think (at least a little) about how you might
  present it.  Further, you should work the problem enough in advance
  of class that you can get help from your peers or from me if
  necessary.
\item {\bf Use board space well.}  Write legibly.  Try not to block
  what you're writing.  Draw appropriate pictures (or refer to figures
  in the text).
\item {\bf Talk to the class, not the board (or the instructor).} This
  doesn't just mean that you ought to face the class when speaking
  most of the time, but that you have a responsibility when you're at
  the front of the room to make sure that everyone is following you.
  If people look puzzled or concerned, ask them what's bothering them.
  You may have made a mistake, or there may be something you can
  elaborate on which will help them understand better.
\item {\bf Articulate the physics present in the problem.}  Some
  problems will have a lot of obvious physics in them, while others
  will appear to be mostly math.  Try to establish some relevance,
  especially for these latter problems.  When might the math be
  useful?  What are the applications?  What is the motivation?
\item {\bf Explain the math.}  Don't just work the math; explain what
  you're doing.  Verbal descriptions of even simple steps can help the
  class follow what you're doing on the board ({\it i.e.,} ``next we
  collect terms\ldots'').  If there is some special technique or trick
  which is important to the problem ({\it e.g.,} ``here is where we
  make use of orthogonality\ldots'') make special note of it.
\item {\bf Ask for questions or other input.}  Soliciting input at the
  end of a problem is always good practice.  It's also helpful to ask
  for questions or comments if a particular step is tricky or gave you
  trouble.
\end{itemize}

\end{document}

%%% Local Variables: 
%%% mode: latex
%%% TeX-master: t
%%% End: 
