\documentclass{article}
\usepackage{charter}
\usepackage{fullpage}
\begin{document}
\centerline{\Large Computational Problems}
\vskip0.1cm\hrule\vskip0.5cm A number of the homework problems
assigned this semester are denoted as computational problems.  As
computers play a continually increasing role in scientific work, it is
important to know under what circumstances and for what kinds of
problems a computer can be a useful tool.  It is unfortunately true
that using a computer inappropriately (either for the wrong type of
problem, using it in the wrong way, or beginning computation at the
wrong point in the solution process) can actually make some problems
much more difficult than they would be if you simply did them by hand.

There are four main classes of computerized tools which are useful in
doing scientific work:
\begin{enumerate}
\item Computer algebra systems (CAS, examples include Maple,
  Mathematica, and Maxima)
\item Matrix processing systems (examples include IDL, Matlab, and
  Octave)
\item General programming languages (most computational scientists use
  Fortran, but many others are available)
\item Typesetting and presentation systems (word processors,
  PowerPoint, TeX, etc.)
\end{enumerate}
In this class, we will deal solely with the first of these classes of
tools, and the tool we will use is Maxima.  Those of you taking
Classical Mechanics this semester will also be exposed to Octave; for
a (very) few of our computational problems you may feel that Octave is
a better fit, and you may use it if you so choose.

If you have access to Maple or Mathematica and are already familiar
with one of those tools, you may find it easier to use what you are
already familar with than to learn a new system.  I encourage you,
however, to learn Maxima for four reasons:
\begin{enumerate}
\item My solutions will be prepared in Maxima, so if you use something
  else you will have to translate the syntax for my solutions to be
  useful to you.
\item My exposure to Maxima is more recent than my exposure to
  Mathematica, so I will be better able to help you with problems in
  your Maxima expressions (I have essentially zero Maple exposure, so
  if you're using that, you're on your own).
\item Maxima is freely available; you can download a copy and install
  it on your own computer, and then take it with you when you leave
  Marietta College.
\item Maxima is installed on computers in Selby 131 (and elsewhere
  around the department) so it should be available to you when needed.
\end{enumerate}

My goal is to have you submit these problems electronically, though as
of this writing I am still trying to figure out the best way to have
you do this. Regardless of how you submit the problems for grading,
you should print out a copy and include it in your homework portfolio.
This means that the solution you generate should conform to the
guidelines for problems in the homework portfolio, namely, it should
have a clear statement of the problem; it should include index terms;
and it needs to include some explanatory text.

In wxMaxima (a GUI frontend for Maxima), you can simply create ``text
cells'', which will give you a place to write your explanatory text.
If you use some other frontend, you will have to include the text in
the form of comments.

More information about Maxima, including a download link and
tutorials, is available at \newline
http://wxmaxima.sourceforge.net \newline  I
strongly recommend that you work through (at least) the 10 minute
Maxima tutorial; it contains nearly everything you will need to get
through the first assignment.

\end{document}
