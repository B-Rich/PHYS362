\documentclass{article}
\usepackage{fullpage}
\usepackage{charter}

\begin{document}
\centerline{\Large The Homework Portfolio}
\vskip0.1cm\hrule\vskip0.5cm

A portion of your {\em homework} grade will come from graded
assignments handed in during each class period.  The other segment will
come from an evaluation of a homework portfolio which you will
assemble over the course of the semester.  The homework portfolio
serves several purposes:
\begin{enumerate}
\item{} It is designed to help you think about how to effectively communicate
the solutions to problems.
\item{} It is designed to allow you to integrate knowledge and understanding
longitudinally across the course of the semester.
\item{} It is designed to be a useful tool for you as you continue to study
physics (and related subjects) into the future.
\end{enumerate}
The basic premise of a homework portfolio is simple: as you turn in
(and I grade and return) homework over the course of the semester, it
accumulates in a single repository which you then can refer back to at
a later time.  However, there are a couple of details which require
additional consideration.  

First, the solutions in your homework portfolio need to be correct
(purpose 3).  Remember, the point is to have a reference you can use
at some point in the future, and an incorrect reference is more
harmful than useful.  This means that when portions of the solutions
you turn in are incorrect (or incomplete) you need to make corrections
and/or additions before putting them into your portfolio.

Second, your solutions need to be self-contained and easy to follow
(purposes 1 and 3).  Please follow these guidelines as you prepare
your homework to go in the portfolio:
\begin{itemize}
\item{\bf Begin each solution with a statement of the problem.} This
  can be briefer than what is given in your textbooks, but should
  include enough information for someone to try to solve the problem
  without having to look it up.
\item{\bf Each solution should begin on its own sheet of paper.}  At
  this level, most problems will be long enough to require a full page
  (or more); you won't be saving much paper by crowding the problems,
  and having each solution start at the top of its own page makes them
  easier to find.  Also, if it turns out that you need to add
  something to the end of a solution, you can do it easily without
  disturbing the beginning of the next problem.
\item{\bf Use only one side of the paper.}  This will improve
  long-term readability of your solutions.  I really don't care what
  kind of paper you use, but have found that for me, white unlined
  paper works best.
\item{\bf Include diagrams/sketches/pictures where possible.}  Note
  that I say ``where possible'' instead of ``where necessary''.  If
  you can possibly include a picture to make things more clear, do so.
\item{\bf Neatness counts!}  I'm not going to require that you learn
  calligraphy to do your homework.  However, it is true that if you
  can't read it, it isn't going to do you any good.  If your
  handwriting is messy, slow down and try to be neater.  This is
  especially true of Greek letters that you may not have had much
  exposure to in the past.  Practice makes better.
\item{\bf Explain more than you think you should.}  One conceptual
  model of homework is that you are trying to write enough to convince
  the grader that you know what you're doing.  That is the wrong
  approach for this class.  Your homework is supposed to be sufficient
  to teach a future you how to do the problem once you have forgotten
  it.  Write enough so that you will be able to follow what's going on
  6 months from now.  Better yet, write the problem so that you would
  have been able to follow it a week or two before it was assigned.
\item{\bf Include references.}  This is especially true if you use an
  example from the text (or a previous homework problem) as part of
  your solution.  It's perfectly fine to say something like ``Boas
  shows on page 103 that...'' rather than showing it again yourself.
  You ought to make a note of anything and everything that helped you
  solve this problem.  This includes coming to office hours or a group
  study session.
\item{\bf Provide large enough margins/whitespace for future notes and
    cross references.}  Your understanding of these problems will
  undoubtedly evolve over he course of the semester.  Give yourself
  room to add new insights to your solutions without having to rewrite
  them.  Leaving room now will save you work later.
\end{itemize}
As you turn in your homework assignments, I will give you feedback on
how to improve your presentation.  Before too long the presentation
issues should become second nature to you, and you won't have to think
about them quite as much.  At each midterm exam, I will evaluate your
homework portfolio and give you feedback using the rubric which will
ultimately be used to grade it (attached to this document).

To accomplish purpose 2, your homework portfolio needs to be
cross-referenced.  You will supplement your individual homework
assignments with a table of contents and an index, as well as marginal
notes directing the reader to other problems.  Each problem should
include an explicit list of index terms either between the statement
of the problem and the solution, or at the end of the solution.  The
index should be an exhaustive list of the index terms from the
solutions, with references back to the appropriate problems.  Index
references should give both a page number within your portfolio as
well as the problem's original source ({\it e.g.}, p. 35; Boas,
8.13.34) In the table of contents, you should include both the
reference to the problem's original source as well as a very brief
(just a few words) summary of the statement of the problem.

{\bf How to use the portfolio:}
When you are assigned a problem, you should begin by (tentatively)
assigning index terms, and then looking at solutions in your portfolio
which share one or more of those terms.  If the solutions are useful
in solving the problem, you should make marginal notes to that effect.
Something like ``Used in solution to Boas, 8.13.34, p.35'', with the
date.  This is true even (or perhaps especially) if it is just a small
portion of the solution which is used.  For example, there may be a
particular mathematical technique or physical idea which crops up more
than once in seemingly unrelated contexts; making a note of this helps
to develop the connections which are vital to a deep understanding of
physics.

The easiest way (in my opinion) to store the homework portfolio is to
keep everything in a three ring binder.  Unlike a lab notebook, which
you want to keep an accurate record of what happened in lab, the
homework portfolio should be a dynamic document which represents your
current state of understanding at any given moment.  The flexible
nature of a three ring binder lends itself to this end nicely.

I will collect your Homework portfolios toward the end of the semester
(see syllabus for date), and return them to you on the last day of
class.  I will grade them based more on presentation than on content
(the content will have been graded when you turned in the assignments
the first time), however you should remember that correctness is one
of the necessary features of a useful solution.  That said, I will
only look at two or three (randomly chosen) solutions in detail when
grading your portfolios.  The remainder of the grade will come from
the usefulness/completeness of the index, table of contents, and cross
references; the presentation quality of the package as a whole; and
evidence that you have used the portfolio throughout the semester.

\end{document}
