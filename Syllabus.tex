\documentclass{article}
\usepackage{charter}
\usepackage{fullpage}
\begin{document}
\centerline{\Large Quantum Mechanics I}
\vskip0.1cm\centerline{\large Marietta College Physics 362, Fall 2016}
\vskip0.1cm\centerline{Dr. Cavendish Q. McKay}
\vskip 0.25cm
\centerline{Syllabus}
\vskip0.1cm\hrule\vskip0.5cm
\begin{description}
\item[Course details:] TTh 1:00 -- 2:15 PM; Rickey 104
\item[Texts:] {\bf Introduction to Quantum Mechanics},
  Griffiths, Pearson/Prentice Hall 2005.
\item[Contact information:] RSC/AHP 143, 376-4871,
  cavendish.mckay@marietta.edu
\item[Office hours:] MWF 9:00--10:00 and 11:00--12:00, or any other time my office
  door is ajar.

\item[What this course is all about:] {\ } \newline In Modern Physics,
  you were exposed to some of the basics of quantum mechanics.  In
  this course, we will revisit many of those ideas with more depth and
  a greater degree of mathematical sophistication. We will begin the
  semester by covering some of the basics about probability which we
  will need to make sense of the standard interpretation of quantum
  mechanics.  Then we will consider what Schr\"odinger's equation
  says about a single particle in a variety of one dimensional
  potentials. Most of the potentials we will look at can be solved
  (with some effort) analytically, but we will also discuss some
  techniques for numerical solutions when the analytic approach fails.

  Once we have looked at these specific cases in one dimension, we
  will prepare to move into the three dimensional world by building a
  more formal, abstract framework within which we can understand
  quantum mechanical principles. (Yay! Formalism! Abstraction!)
  Finally, we will study the hydrogen atom in three dimensions.  The
  spherical symmetry of the hydrogen atom will lead us naturally into
  a discussion of angular momentum.

  Throughout the semester, we will make use of computational tools in
  solving problems. The focus is not intended to be on programming as
  a discipline; you should instead think of the computer as a
  sophisticated calculator with a big keyboard. Part of the goal of
  this approach is to help you distinguish between the cases where the
  the use of a computer can be very helpful, marginally helpful, not
  particularly helpful, or actively harmful.
  
\item[Outcomes:] By the end of this course, students should be able to
  \begin{itemize}
  \item Prepare a useful and correct solution to any of the quantum
    mechanical problems we have covered over the course of the
    semester.
  \item Lead a discussion about a homework problem at the board.
  \item Use sympy or a similar computer algebra system in solving
    physics problems.
  \item Be able to judge when it is appropriate to use a computer in
    solving a physics problem.
\end{itemize}
\item[Evaluation:] Homework, exams, and class discussion will
  contribute to your grade.  Each student will have the opportunity
  several times during the semester to lead the class in discussing a
  problem.  Your performance in leading these discussions will be
  graded according to a rubric explained in detail in a separate
  handout, and will account for 15\% of your final grade. There will
  be two midterm exams (each worth 15\% of your final grade) and a
  final exam, worth 20\%.  Homework problems will be due at the
  beginning of each class period.  When these problems are returned to
  you, you will assemble them in a homework portfolio (explained in
  detail in a separate handout) which will be graded at the end of the
  semester.  The homework problems count for 20\% of your final grade;
  the portfolio is an additional 15\%.  You will be able to use your
  portfolio (and only your portfolio) as a reference during exams.

  Some of the assigned problems are marked as ``computational''
  problems.  You are expected to use sympy (a freely available python library for doing algebra and calculus on a computer) to write
  your solutions to these problems.  I will spend some class time on
  teaching you to use sympy, and will also point you toward some
  online resources you can use to learn more. I will distribute to you
  all of my solutions to the computational problems (after you have
  turned them in, of course).

  Summarizing:
  \begin{tabular}{lcr}
    Leading discussion of board problems & 15\% &\\
    Homework problems & 20\% &\\
    Homework Portfolio & 15\% &\\
    Midterm 1& 15\% & \\
    Midterm 2& 15\% & \\
    Final Exam & 20\% & Friday, May 1, 8:30 AM\\
  \end{tabular}

\item[Policies:] {\ } \newline
  \begin{itemize}
  \item I will endeavor to return homework to you the class period
    after you turn it in to me.  Therefore, no late homework will be
    accepted.
  \item Exams must be taken at the scheduled time unless a documented
    excuse is presented.  If possible, arrangements for a make-up exam
    should be made prior to the scheduled exam time.
  \item Students who believe that they may need accommodations due to
    a documented disability should contact the Academic Resource
    Center (Andrews Hall, Third floor, 376-4700) and the instructor as
    soon as possible to ensure that such accommodations are
    implemented in a timely manner. You must meet with the ARC staff
    to verify your eligibility for any accommodation and for academic
    assistance.
  \item The following statement is an excerpt from the {\bf Marietta
    College Undergraduate Programs, 2010-2011 Catalog}, page 121:
  \begin{quotation} Dishonesty within the academic community is a very
    seriouss matter, because dishonesty destroys the basic trust
    necessary for a healthy educational environment. Academic
    dishonesty is any treatment or representation of work as if one
    were fully responsible for it, when it is in fact the work of
    another person.  Academic dishonesty includes cheating,
    plagiarism, theft, or improper manipulation of laboratory or
    research data or theft of services. A substantiated case of
    academic dishonesty may result in disciplinary action, including a
    failing grade on the project, a failing grade in the course, or
    expulsion from the College.\end{quotation}
  \item I reserve the right to adjust this syllabus should it become
    necessary.
\end{itemize}
\end{description}
\newpage
\hspace*{-1cm}\begin{minipage}{\textwidth}
\centerline{Approximate schedule}
\vskip0.1cm\hrule\vskip0.5cm
  \begin{tabular}{llr}
    Date & Topic & Assigned problems ({\underline{underlined}} problems are computational)\\[0.5ex]

T 30 Aug & Introduction, Probability& \\
Th 1 Sep & Normalization, Uncertainty & 1.1, {\underline{1.3}}, 1.4, 1.11 {\underline{1.5}}, 1.6\\

T 6 Sep & Computational issues, Schr\"odinger equation & 1.7, {\underline{1.9}}, 1.12, 1.15, 1.16,\\
Th 8 Sep & Infinite Square Well & {\underline{1.17}}, 2.1, 2.2, 2.3\\\

T 13 Sep & &  {\underline{2.4}}, 2.5, 2.7, 2.8, 2.38\\
Th 15 Sep & Harmonic Oscillator & 2.39,  2.10, {\underline{2.11}}, 2.12\\

T 20 Sep & Free Particle &2.13, {\underline{2.15}}, {\underline{2.17}}, 2.21 \\
Th 22 Sep & Delta Function Potential & 2.23, 2.24, 2.27\\

T 27 Sep & Finite Square Well& 2.20, 2.26,2.29, {\underline{2.30}} \\
Th 29 Sep & Shooting & 2.34, 2.35\\

T 4 Oct & Review &  {\underline{2.54}}, {\underline{2.55}}, {\underline{2.56}}\\
Th 6 Oct & Midterm Exam I& Portfolio check\\

T 11 Oct & Fall Break, No class& \\
Th 13 Oct & Hilbert Space & \\

T 18 Oct &Observables &3.1, 3.2\ \\
Th 20 Oct & &  3.3, 3.4, 3.5\\

T 25 Oct & Eigenfunctions of Hermitian operators &3.6, 3.7, 3.8 \\
Th 27 Oct & & 3.9, 3.10, {\underline{3.11}}\\

T 1 Nov & Uncertainty & 3.13, 3.14, 3.15, 3.17, 3.31\\
Th 3 Nov & Dirac Notation &  3.27, 3.21, 3.22, 3.23, 3.24\\

T 8 Nov & Review & 3.33, 3.34\\
Th 10 Nov &Midterm Exam II & Portfolio Check\\

T 15 Nov & Separation of Variables! &4.1, 4.2 \\
Th 17 Nov & & {\underline{4.3}}, {\underline{4.4}}, {\underline{4.5}} \\

T 22 Nov & Hydrogen & 4.6, 4.9, {\underline{4.10}}, {\underline{4.11}}\\
Th 24 Nov & & Thanksgiving, no class\\

T 29 Nov & &{\underline{4.12}}, 4.13, 4.14, {\underline{4.15}}\\
Th 1 Dec & Angular Momentum & 4.16, 4.17, 4.18, 4.19\\

T 6 Dec & Spin &4.21, 4.22, {\underline{4.23}},4.27, 4.28, \\
Th 8 Dec & & 4.29, 4.31, 4.34, 4.35\\

F 16 Dec& Final Exam & 3:00--5:30
\end{tabular}
\end{minipage}

\end{document}

%%% Local Variables: 
%%% mode: latex
%%% TeX-master: t
%%% End: 
